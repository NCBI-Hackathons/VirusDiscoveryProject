\section{Introduction}

While advances in sequencing technology have greatly reduced the cost of whole
genome sequencing \cite{Mardis2011}, it has given rise to new problems, especially related to
data analysis and management. As the number of bases in the Sequence Read
Archive \cite{Kodama2012} exceeds 33 petabases (June 2019), the difficulty to navigate and
analyze all of this data has grown as well. Furthermore, as the number of data
warehouses grows with the increased accessibility of the technology, the need
to support interoperability of data types increases. To address these issues,
the National Institutes of Health (NIH) launched the Science and Technology
Research Infrastructure for Discovery, Experimentation, and Sustainability
(STRIDES, \cite{StridesWeb}) initiative. "Through the STRIDES Initiative partnerships, NIH
will test and assess models of cloud infrastructure for NIH-funded datasets and
repositories."

As part of the initiative, the National Center for Biotechnology Information
(NCBI) \cite{Sayers2019} launched a series of hackathons (see \url{biohackathons.github.io}), the
first of which, the Virus Discovery hackathon, was held in January 2019 in San
Diego. These events gather researchers for three days to work on projects
around a topic, and provide an opportunity to quickly proptype a solution or a
set of tools to address a community need. NCBI hackathons also facilitate
networking among researchers, and allow NCBI staff to identify opportunities to
improve their services. While previous hackathons were not specifically focused
on working with large volumes of data or compute intensive tasks in the cloud,
they provide a framework with which to engage the research community. As part
of the STRIDES Initiative the hackathons are particularly focused on allowing
researchers to work with large amounts of data in a cloud environment, in an
effort to identify the needs and challenges of this new research environment.
Typically topics are developed in conjunction with a host researcher and
assigned to different working groups. Team leaders of these working groups are
consulted to further refine the topics after some initial recruitment. On the
first day of the hackathon, the teams further scope directions, and then
continue to iterate on development goals over the course of the three days.
Additionally, one writer from each group participates in a break-out session
each day to help guide documentation of the work done.

While there are a number of commercial cloud providers, including Amazon
(\url{https://aws.amazon.com}), Microsoft (\url{https://azure.microsoft.com}),
 and Google (\url{https://cloud.google.com}), at the time of this hackathon an agreement had
been reached with Google as part of STRIDES. Google's cloud platform offers
scalable nodes, with highly configurable access-control settings as well as an
SQL-like database infrastructure, BigQuery. Details on cloud infrastructure can
be reviewed here. Briefly, cloud compute refers to remotely hosted computers,
some parts of which can dynamically access a single compute instance. This
access to computing allows research scientists and organizations to use large
and scalable computing resources without investing in the required
infrastructure and maintenance. While this lowers the barrier to access
supercomputer-type resources, it does not provide a comprehensive solution to
the general scientific public. The main barriers to adoption by researchers
include 1) modest experience in UNIX command-line type environments, and 2) the
ineffectiveness of most commonly used bioinformatics tools to leverage the
compute resources available in a cloud-computing setting. STRIDES hackathons
aim to identify to what extent these barriers impact working researchers, and
how they would like them addressed.

One of the fastest growing sources of public biological data is Next Generation
Sequencing (NGS) data, housed in NCBI's Sequence Read Archive (SRA)
\cite{Leinonen2010}. SRA includes results from amplicon and whole genome
shot-gun studies, conducted on a variety of sequencing platforms. The data is
derived from research in many fields such as personalized medicine, disease
diagnosis, viral or bacterial evolution, sequencing efforts targeting
endangered animals, and sequencing of economically significant crops, among
others. Despite the scientific potential of these datasets, there are several
impediments to their usage. For one, sample metadata standards can vary between
studies, making it difficult to identify datasets based on a common set of
descriptive sample attribute terms. Moreover, while the content of some
datasets is explicit, this is often not the case, particularly in those samples
derived from complex mixtures of organisms like those from human gut  and
environmental samples.  In these cases, actual organismal content may not be
known, either because it was never fully evaluated or because the content
includes novel organisms not described in reference sets or otherwise
undocumented, i.e. the so-called “dark matter” \cite{Roux2015}.

Understanding the microbial composition of different environments is necessary
to support comparisons between samples and to establish relationships between
genetics and biological phenomena. If such information was available for all
SRA datasets, it would greatly improve both findability of specific datasets
and the quality of analysis that could be conducted. However, determining the
organismal content of a sample is not always an easy task as it typically
requires comparisons to existing genome references. This can be difficult when
samples include viruses because only a small portion of Earth’s viral diversity
has been identified and made available in reference sets \cite{Carroll2018}.
Even when content can be identified, the very large size of SRA datasets
present a significant scalability problem, and strategies must be developed to
support large scale organismal content analysis in order to provide an index of
this content for use in data search and retrieval. To that end, the first
STRIDES hackathon engaged researchers working in this field to leverage the
computational power of the Google cloud environment and test the applicability
of several bioinformatic approaches to the identification of both known and
novel viruses in existing, public SRA datasets

Here are presented the general results of these efforts, with an emphasis on
challenges participants faced in conducting their work. Firstly, the scientific
staff involved in the hackathon is presented, including their demographics and
research backgrounds. Scientists were organized in teams which roughly
corresponds to the different research sections found in this article, these
include: data selection, taxonomic and cluster identification, domain and gene
annotation.
